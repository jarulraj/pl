
\documentclass{article}

\usepackage[usenames,dvipsnames]{xcolor}
\usepackage{amsmath}
\usepackage{latexsym,amsthm,amssymb,amscd,url,enumerate}
\usepackage[show]{ed}
\usepackage[all]{xy}
\usepackage{tikz}
\usepackage{tikz-cd}
\usepackage{leftidx}
\usetikzlibrary{arrows}
\usepackage{amsmath}
\usepackage{amssymb}
\usepackage{amsthm}
\usepackage{stmaryrd}
\usepackage{wasysym}
\usepackage{proof}
\usepackage{hyperref}
\usepackage{framed}
\usepackage{fullpage}

\newtheorem{theorem}{Theorem}						
\newtheorem{definition}[theorem]{Definition}
\newtheorem{proposition}[theorem]{Proposition}
\newtheorem{lemma}[theorem]{Lemma}
\newtheorem{axiom}{Axiom}
\newtheorem{notation}[theorem]{Notation}
\newtheorem{corollary}[theorem]{Corollary}

%% ==================================================================
%% MACROS
%% ==================================================================

\newcommand{\E}{\mathtt{E}}
\newcommand{\B}{\mathtt{B}}
\newcommand{\C}{\mathtt{C}}
\newcommand{\R}{\mathtt{R}}
\newcommand{\N}{\mathtt{N}}
\newcommand{\LL}{\mathtt{L}}
\newcommand{\true}{\mathtt{true}}
\newcommand{\false}{\mathtt{false}}
\newcommand{\andsym}{\mathtt{and}}
\newcommand{\orsym}{\mathtt{or}}
\newcommand{\notsym}{\mathop{\mathtt{not}}}
\newcommand{\ifsym}{\mathtt{if}}
\newcommand{\then}{\mathtt{then}}
\newcommand{\elsesym}{\mathtt{else}}
\newcommand{\whilesym}{\mathtt{while}}
\newcommand{\dosym}{\mathtt{do}}
\newcommand{\skipsym}{\mathtt{skip}}
\newcommand{\nil}{\mathtt{nil}}
\newcommand{\case}{\mathtt{case}}
\newcommand{\digit}{\mathtt{d}}
\newcommand{\negation}{\mathtt{neg}}
\newcommand{\Digit}{\mathbf{Digit}}
\newcommand{\denot}[1]{\mathtt{[[{#1}]]}}
\newcommand{\Sem}{\mathtt{S}}

\newcommand{\G}{\Gamma}
\newcommand{\D}{\Delta}

\newcommand{\type}{\;\mathsf{type}}
\newcommand{\val}{\;\mathsf{val}}
\newcommand{\dom}[1]{\mathsf{dom}(#1)}
\newcommand{\FV}{\mathsf{FV}}
\newcommand{\reduces}{\mapsto}

\newcommand{\ctx}[3]{(#1,#2\rhd #3)}
\newcommand{\bool}{\mathbf{2}}
\newcommand{\boolt}{\mathsf{tt}}
\newcommand{\boolf}{\mathsf{ff}}
\newcommand{\sem}[1]{\llbracket #1 \rrbracket}

% Logical equivalence
\newcommand{\rels}[3]{#1:#2\leftrightarrow #3}
\newcommand{\relto}{\hookrightarrow}
\newcommand{\LR}[2]{\sem{#1}_{#2}}

\newcommand{\gd}[1]{\hat\gamma(\hat\delta(#1))}
\newcommand{\gdp}[1]{\widehat{\gamma'}(\widehat{\delta'}(#1))}

% Existential types
\newcommand{\pack}[3]{\mathsf{pack}\,#1\,\mathsf{with}\,#2\,\mathsf{as}\,#3}
\newcommand{\open}[4]{\mathsf{open}\,#1\,\mathsf{as}\,#2\,\mathsf{with}\,#3\,\mathsf{in}\,#4}

% Data types
\newcommand{\Nat}{\mathsf{nat}}
\newcommand{\z}{\mathsf{z}}
\newcommand{\suc}{\mathsf{suc}}
\newcommand{\natrec}{\mathsf{natrec}}
\newcommand{\pred}{\mathsf{pred}}
\newcommand{\sub}{\mathsf{sub}}

\newcommand{\List}{\mathsf{list}}
\newcommand{\cons}{\mathsf{cons}}
\newcommand{\foldr}{\mathsf{foldr}}

\newcommand{\fst}{\mathsf{fst}}
\newcommand{\snd}{\mathsf{snd}}

\newcommand{\oList}{\overline{\mathsf{list}}}
\newcommand{\onil}{\overline{\mathsf{nil}}}
\newcommand{\ocons}{\overline{\mathsf{cons}}}
\newcommand{\ohead}{\overline{\mathsf{head}}}
\newcommand{\otail}{\overline{\mathsf{tail}}}
\newcommand{\ofoldr}{\overline{\mathsf{foldr}}}

% crazy hacks
\makeatletter
\def\lam#1{{\lambda}\@lamarg#1:\@endlamarg\@ifnextchar\bgroup{.\,\lam}{.\,}}
\def\@lamarg#1:#2\@endlamarg{\if\relax\detokenize{#2}\relax #1\else\@lamvar{\@lameatcolon#2},#1\@endlamvar\fi}
\def\@lamvar#1,#2\@endlamvar{#2\,{:}\,#1}
\def\@lameatcolon#1:{#1}
\let\lamt\lam
\makeatother

% disable hbox warnings
\hfuzz=5.002pt 

%% ==================================================================
%% HELPERS
%% ==================================================================

\definecolor{DarkBlue}{rgb}{0.00,0.30,0.90}

\newcommand{\question}[1]
{\color{DarkBlue}#1 \color{Black}}

\definecolor{comment-color}{rgb}{1,0,0}
\renewcommand{\todo}[1]{
\textnormal{\color{comment-color}{\textbf{TODO: #1}}}\unskip}

%% ==================================================================
%% DOCUMENT
%% ==================================================================

\begin{document}

\title{\textbf{15-812 Semantics of Programming Languages \\ Spring 2015 \\
{\large Assignment 2 Solutions}}\\\vspace{0.3in}
{\Large \bf Joy Arulraj (AID : jarulraj)}}

\date{}
\author{}

\maketitle

\question{
\noindent In this assignment we will revisit the imperative language $\mathcal{L}$ defined by the syntax
\begin{align*}
& \E := \mathtt{n} \; | \; \mathtt{x} \; | \; \E_1 + \E_2 \; | \; \E_1 \cdot \E_2 \; | \; \negation(\E) \; | \; \ifsym \; \B \; \then \; \E_1 \; \elsesym \; \E_2 \\ 
& \B := \true \; | \; \false \; | \; \B_1 \; \andsym \; \B_2 \; | \; \B_1 \; \orsym \; \B_2 \; | \; \notsym \; \B \; | \; \E_1 \leq \E_2 \\
& \C := \skipsym \; | \; \mathtt{x} := \E \; | \; \C_1 ; \C_2 \; | \; \ifsym \; \B \; \then \; \C_1 \; \elsesym \; \C_2 \; | \; \whilesym \; \B \; \dosym \; \C
\end{align*}
and the state-transition $[[\cdot]]$, input/output $\mathcal{I}$, and termination $\mathcal{T}$ semantics for $\mathcal{L}$ as defined in class. Recall that two programs $\C_1, \C_2$ are 

\begin{enumerate}

\item \emph{equivalent} with respect to a semantics $\Sem$ 
if $\Sem(\C_1) = \Sem(\C_2)$
\item \emph{contextually equivalent} with respect to $\Sem$ 
if $\forall \mathcal{C}, \Sem(\mathcal{C}[\C_1]) = \Sem(\mathcal{C}[\C_2])$, where $\mathcal{C}$ ranges over program contexts

\end{enumerate}

For two semantics $\Sem_1, \Sem_2$ we say that
\begin{enumerate}
	\item $\Sem_1$ is \emph{compositional} with respect to $\Sem_2$ if for any two programs $\C_1,\C_2$, 
	we have that $\Sem_1(\C_1) = \Sem_1(\C_2)$ implies $\forall \mathcal{C}, \Sem_2(\mathcal{C}[\C_1]) = \Sem_2(\mathcal{C}[\C_2])$

	\item $\Sem_1$ is \emph{fully abstract} with respect to $\Sem_2$ if for any two programs $\C_1,\C_2$, 
	we have that $\forall \mathcal{C}, \Sem_2(\mathcal{C}[\C_1]) = \Sem_2(\mathcal{C}[\C_2])$ implies $\Sem_1(\C_1) = \Sem_1(\C_2)$
	\end{enumerate}
	A semantics $\Sem$ is \emph{compositional} if it is compositional with respect to itself.\bigskip

\noindent \textbf{Please justify all answers.}
}

%% ==================================================================
%% REVISITING SEMANTICS
%% ==================================================================

\section{Revisiting semantics}

\begin{enumerate}
\question{
\item[1.1] 
\begin{enumerate}
	\item[a)] Is the input/output semantics $\mathcal{I}$ compositional? Is it
	fully abstract with respect to the termination semantics $\mathcal{T}$?
	\item[b)] Is the termination semantics $\mathcal{T}$ compositional? Is it
	fully abstract with respect to the input/output semantics $\mathcal{I}$?
\end{enumerate}
}

\begin{enumerate}
\item[a)] \begin{itemize}

\item \textbf{Compositionality property :}\\

	\textbf{Yes}, $\mathcal{I}$ is compositional with respect to itself.
	For this to hold, we need to show that :
	
	$\mathcal{I}(\C_1) = \mathcal{I}(\C_2) \rightarrow 
	\forall \mathcal{C}, \;	
	\mathcal{I}(\mathcal{C}[\C_1]) = \mathcal{I}(\mathcal{C}[\C_2])$.
		
	By definition, \\
	
	$\mathcal{I} : \textbf{Com} \rightarrow (St \rightarrow St_{ \bot })$\\
	$ \mathcal{I}(c) = | c | \cup  \{ (\sigma, \bot) \; | \; \sigma
	\notin dom | c | \} $\\
		
	For any two programs, $\C_1,\C_2$, if we have that $\mathcal{I}(\C_1) =
	\mathcal{I}(\C_2)$, 
	then this implies that $|\C_1| = | \C_2|$. This follows directly from the
	definition.	
	Thus, $\mathcal{I}$ semantics and $[[\cdot]]$ semantics are
	equipollent.
	Due to the compositionality of the state transition semantics,
	$|\C_1| = | \C_2|$ if and only if 
	$\forall \mathcal{C}, \; \mathcal{I}(\mathcal{C}[\C_1]) =
	\mathcal{I}(\mathcal{C}[\C_2])$.  \\
	
	Hence, $\mathcal{I}$ is compositional.\\

\item \textbf{$\mathcal{I}$ is fully abstract with respect to the termination
semantics $\mathcal{T}$ :}\\

	\textbf{Yes}, $\mathcal{I}$ is fully abstract with respect to the termination
	semantics $\mathcal{T}$.\\

	For this property to hold, we need to show that :
	
	$\forall \mathcal{C}, \;	
	\mathcal{T}(\mathcal{C}[\C_1]) = \mathcal{T}(\mathcal{C}[\C_2])	
	\rightarrow 
	\mathcal{I}(\C_1) = \mathcal{I}(\C_2) $ \\
	
	We prove this by proving its contrapositive. 

	$ \mathcal{I}(\C_1) \neq \mathcal{I}(\C_2) 	\rightarrow 	
	\exists \; \mathcal{C},	
	\mathcal{T}(\mathcal{C}[\C_1]) \neq \mathcal{T}(\mathcal{C}[\C_2])$ \\

 	Consider a set of commands $\C_1$ and $\C_2$, where $\mathcal{I}(\C_1) ̸\neq
 	\mathcal{I}(\C_2)$.
	We will show that there exists a context $C$ in which 
	$\mathcal{T}(C [\C_1 ]) \neq \mathcal{T}(C [\C_2 ])$.
	Let the state transition $(\sigma,\tau) \in \mathcal{I}(\C_1)$ and
	$(\sigma,\tau) \notin \mathcal{I}(\C_2)$.
	Let the $dom(\tau)$ comprise of $x_{l}$ and let $\tau(x_{l}) = v_{l}$. 
	Let $b$ be the boolean expression $x_{l} = v_{l}$.\\

	Now, consider the context $\mathcal{C} := [ \-- ];$ $\ifsym$ b $ \then \;
	\skipsym \; \elsesym \; \whilesym \; \true \; \dosym \; \skipsym$. Clearly, executing
	$\mathcal{C}[\C_1]$ from $\sigma$ terminates, but executing $\mathcal{C}[\C_2]$
	from $\sigma$ does not terminate.
	Thus, $\mathcal{T}(C [\C_1 ]) \neq \mathcal{T}(C [\C_2 ])$.
	We have shown that contextual equivalence with respect 
	to $\mathcal{I}$ is the same as contextual equivalence with respect to
	$\mathcal{T}$.
	Therefore, $\mathcal{I}$ is fully abstract with respect to the termination
	semantics $\mathcal{T}$.\\
			
\end{itemize}

\item[b)] \begin{itemize}

\item \textbf{Compositionality property :}\\

	\textbf{No}, $\mathcal{T}$ is not compositional with respect to itself.
	For $\mathcal{T}$ to satisfy the compositionality property, 
	we need to show that :\\
	
	$\mathcal{T}(\C_1) = \mathcal{T}(\C_2) \rightarrow 
	\forall \mathcal{C}, \;	
	\mathcal{T}(\mathcal{C}[\C_1]) = \mathcal{T}(\mathcal{C}[\C_2])$.\\

	We show a counterexample to disprove this. Consider	any two programs,
	$\C_1,\C_2$, such that $\mathcal{T}(\C_1) = \mathcal{T}(\C_2)$.
	We show that there exists a context in $\mathcal{C}$, such that
	$\mathcal{T}(\mathcal{C}[\C_1]) \neq \mathcal{T}(\mathcal{C}[\C_2])$.\\
			
	Consider the programs $\C_1 = x := 1; $ and $\C_2 = x := 2;$.\\
	$\mathcal{T}(\C_1) = \{ (\sigma, \top) \; | \; x \in dom(\sigma) \} 
	\cup \{ (\sigma, \bot) \; | \; x \notin dom(\sigma) \}$ and\\ 
	$\mathcal{T}(\C_2) = \{ (\sigma, \top) \; | \; x \in dom(\sigma) \} 
	\cup \{ (\sigma, \bot) \; | \; x \notin dom(\sigma) \}$. 
	Thus, $\mathcal{T}(\C_1) = \mathcal{T}(\C_2) $.\\
	
	However, consider the context $C = [ \-- ];$ $\ifsym \; x = 1 \; \then \;
	\skipsym \; \elsesym \; \whilesym \; \true \; \dosym \; \skipsym$.
	In this case,  $\mathcal{T}(\mathcal{C}[\C_1]) \neq \mathcal{T}(\mathcal{C}
	[\C_2])$, because it terminates in the former case but not in the later
	case. Hence, there exists a context in $\mathcal{C}$, such that
	$\mathcal{T}(\mathcal{C}[\C_1]) \neq \mathcal{T}(\mathcal{C}[\C_2])$.
	Therefore, $\mathcal{T}$ is not compositional. \\

\item \textbf{$\mathcal{T}$ is fully abstract with respect to the termination
semantics $\mathcal{I}$ :}\\

	\textbf{Yes}, $\mathcal{T}$ is fully abstract with respect to the termination
	semantics $\mathcal{I}$.\\

	For this property to hold, we need to show that :
	
	$\forall \mathcal{C}, \;	
	\mathcal{I}(\mathcal{C}[\C_1]) = \mathcal{I}(\mathcal{C}[\C_2])	
	\rightarrow 
	\mathcal{T}(\C_1) = \mathcal{T}(\C_2) $ \\

    We already showed that $\mathcal{I}$ semantics and $[[\cdot]]$
    semantics are equipollent.
    This implies that if $\forall \mathcal{C}, 
    \mathcal{I}(\mathcal{C}[\C_1]) = \mathcal{I}(\mathcal{C}[\C_2])$, 
    then $| \C_1 | = | \C_2 |$.\\ 

    Further, we showed that contextual equivalence with respect to $\mathcal{I}$
	is the same as contextual equivalence with respect to $\mathcal{T}$.
	Since $[[\cdot]]$ semantics is fully abstract with respect to $\mathcal{I}$, 
	it is also fully abstract with respect to $\mathcal{T}$.
	Thus, if $| \C_1 | = | \C_2 |$, then we have
	$\forall \mathcal{C},	
	\mathcal{T}(\mathcal{C}[\C_1]) = \mathcal{T}(\mathcal{C}[\C_2])$.\\	
    Now, consider the identity context $\mathcal{C} = [ \-- ];$. Clearly,
    if $| \C_1 | = | \C_2 |$, then $\mathcal{T}(\C_1) = \mathcal{T}(\C_2)$. 
    Therefore,  $\mathcal{T}$ is fully abstract with respect to the termination
	semantics $\mathcal{I}$.\\
	
\end{itemize}
\end{enumerate}

\question{	
\item[1.2]
\begin{enumerate}
	\item[a)] Give a semantics $\Sem$ such that the state transition semantics 
	$[[\cdot]]$ is not compositional with respect to $\Sem$.
	\item[b)] Give a semantics $\Sem$ such that the state transition semantics 
	$[[\cdot]]$ is not fully abstract with respect to $\Sem$.
\end{enumerate}
}

\begin{enumerate}
\item[a)] 
	For the $[[\cdot]]$ semantics to not satisfy the compositionality property
	with respect to $\mathcal{S}$, we need to show that :\\
	
	$[[\C_1]] = [[\C_2]] \nrightarrow 
	\forall \mathcal{C}, \;	
	\mathcal{S}(\mathcal{C}[\C_1]) = \mathcal{S}(\mathcal{C}[\C_2])$.\\

	Consider the semantics $\mathcal{DEPTH}$, defined thus :\\

	$\mathcal{DEPTH} : \textbf{Com} \rightarrow (St \rightarrow St)$
	
	$ \mathcal{DEPTH}(c) = \{ (\sigma, [ \sigma | x : depth(c) ] ) 
	\; | \; x \in dom (\sigma) \}$,
	
	where the function depth is defined thus :	
	$depth$ : $\textbf{Com} \rightarrow \N$ 	
	\begin{align*}
	& depth(\skipsym) = 0 \\
	& depth(x := e) = 1 + depth(e) \\
	& depth (\C_1 ; \C_2 ) = 1 + max (depth (\C_1 ), depth (\C_2 ))	\\
	& depth(\ifsym \; b \;  \then \;  \C_1 \;  \elsesym \;  \C_2 ) 
	= 1 + max(depth(b), depth(\C_1), depth(\C_2)) \\
	& depth (\whilesym \; b \;  \dosym \;  \C) = 1 + max (depth (b), depth (\C)) \\
	\end{align*}
	
	Now, consider the commands $\C_1 = x := 2;$ and $\C_2 = x := 1; x := 2;$.
	By definition of the state transition semantics, we observe that 
	$[[ \C_1 ]] = [[ \C_2 ]] $. Intuitively, they both transition from the
	initial state $\sigma$  to a state, where if x $\in dom(\sigma)$, then $x =
	2$.\\
	
	Now, we note that $depth(\C_1) = 1$ and $depth(\C_2) = 2$. Thus,
	$\mathcal{DEPTH}([\C_1]) \neq \mathcal{DEPTH}([\C_2])$, as the values
	of all variables differ in both cases. We have shown that 
	$[[ \C_1 ]] = [[ \C_2 ]]$ does \textit{not} imply
	$\mathcal{DEPTH}(\mathcal{C}[\C_1]) = \mathcal{DEPTH}(\mathcal{C}[\C_2])$,
	in this case for the identity context $\mathcal{C} = [\--];$.
	Thus, $[[\cdot]]$ is not compositional with respect to $\mathcal{DEPTH}$.\\	
 

\item[b)] 
	For the $[[\cdot]]$ semantics to not be fully abstract
	with respect to $\mathcal{S}$, we need to show that :\\

	$\forall \mathcal{C}, \;	
	\mathcal{S}(\mathcal{C}[\C_1]) = \mathcal{S}(\mathcal{C}[\C_2])	
	\nrightarrow 
	[[\C_1]] = [[\C_2]] $ \\

	Consider the semantics $\mathcal{HITCHHIKER}$, defined thus :\\

	$ \mathcal{HITCHHIKER}(c) = \{ (\sigma, [ \sigma | x : 42 ] ) 
	\; | \; x \in dom (\sigma) \}$.\\
	
	We show this using a counterexample. 
	Consider any two programs $\C_1$ and $\C_2$ such that $[[ \C_1 ]] \neq [[\C_2
	]]$, for instance $\C_1 = x := 1;$ and $\C_2 = x := 2;$.
	Let the initial state $\sigma \in dom(\C_1)$ and  $\sigma \in dom(\C_2)$.\\

	We observe that	$\forall \mathcal{C} \;$, 
	$\mathcal{HITCHHIKER}(\mathcal{C}[\C_1]) =
	\mathcal{HITCHHIKER}(\mathcal{C}[\C_2])$.
    This is because, they both transition from the initial state $\sigma$
    to a state, where if x $\in dom(\sigma)$, then $x = 42$.      
    However, $[[ \C_1 ]] \neq [[ \C_2 ]]$. Thus,\\

    $\forall \mathcal{C}, \;	
	\mathcal{S}(\mathcal{C}[\C_1]) = \mathcal{S}(\mathcal{C}[\C_2])	
	\nrightarrow 
	[[\C_1]] = [[\C_2]] $ \\
    
    Hence, $[[\cdot]]$ is not fully abstract with respect to
    $\mathcal{HITCHHIKER}$.

\end{enumerate}
            
\question{
\item[1.3] In the previous homework assignment, question 2, you defined a denotational 
semantics $\Sem$ for the extended imperative language $\mathcal{L}_e$:

\begin{align*}
& \E := \mathtt{n} \; | \; \mathtt{x} \; | \; \E_1 + \E_2 \; | \; \E_1 \cdot \E_2 \; | \; \negation(\E) \; | \; \ifsym \; \B \; \then \; \E_1 \; \elsesym \; \E_2 \\ 
& \B := \true \; | \; \false \; | \; \B_1 \; \andsym \; \B_2 \; | \; \B_1 \; \orsym \; \B_2 \; | \; \notsym \; \B \; | \; \E_1 \leq \E_2 \\
& {\color{red} \LL := \nil \; | \; (\mathtt{n},\C)::\LL} \\
& \C := \skipsym \; | \; \mathtt{x} := \E \; | \; \C_1 ; \C_2 \; | \; \ifsym \; \B \; \then \; \C_1 \; \elsesym \; \C_2 \; | \; \whilesym \; \B \; \dosym \; \C \; | \; {\color{red} \case \; \E \; \LL \; \C}
\end{align*}

Is $\Sem$ compositional? 
Is the state transition semantics $[[\cdot]]$ compositional with respect to $\Sem$? 
Is it fully abstract?
}

The semantic function $\mathcal{S}$ : St $\rightarrow$ St is given by :
\begin{align*}
& S( \; skip \; ) \; = \; \{ (\sigma, \sigma) \; | \; \sigma \in St \} \\
& S( \; x := e \; ) \; = \; \{ (\sigma, [\sigma | x : v]) \; | \; (\sigma, v)
\in | e | \; \& \; x \in dom(\sigma) \} \\
& S( \; \C_1 ; \C_2 ) \;  = \; \{ (\sigma, \sigma'') \; | \; \exists
\sigma'.(\sigma, \sigma') \in | \C_1 | \; \&  \; (\sigma',\sigma'') \in | \C_2 | \} 
\; = \; | \C_2 | \circ | \C_1 | \\
& S( \; \ifsym \; b \; \then  \; \C_1 \; \elsesym \; \C_2 \; ) \; = \; | \;
\C_1 | \; \circ  \; | b |_{\true} \; \cup \; |\C_2| \; \circ \; | b |_{\false} \\
& S( \; \whilesym \; b \; \dosym \; \C \; ) \; = \; | b |_{\false} \; \circ \;
(|c| \; \circ \; | b |_{true} )^{*} \\
& S( \; \case \; \E \; \LL \; \C \; ) \; = \C \; \circ \; | |\E| \in
dom(\LL) ) |_{\false} \cup \; \LL(|\E|) \; \circ \; | |\E| \in dom(L) |_{\true} 
\end{align*}

\begin{itemize}
  
\item We observe that for all commands $\C_1$ and $\C_2$, 
$[[\C_1]] = [[\C_2]]$ if and only if $\mathcal{S}(\C_1) = \mathcal{S}(\C_2)$. 
Hence, $\mathcal{S}$ and $[[\cdot]]$ are 
equipollent. By the compositionality property of $[[\cdot]]$,\\

$[[\C_1]] = [[\C_2]] \rightarrow  
 \forall \mathcal{C}, \;	
 \mathcal{S}(\mathcal{C}[\C_1]) = \mathcal{S}(\mathcal{C}[\C_2])$. \\

Hence, $[[\cdot]]$ is compositional with respect to $S$.

\item We showed that $[[\C_1]] = [[\C_2]]$ if and only if $\mathcal{S}(\C_1) =
\mathcal{S}(\C_2)$, and that 
$[[\C_1]] = [[\C_2]] \rightarrow  \forall
\mathcal{C}, \; \mathcal{S}(\mathcal{C}[\C_1]) = \mathcal{S}(\mathcal{C}[\C_2])$. \\

 Thus,  $\mathcal{S}(\C_1) = \mathcal{S}(\C_2) \rightarrow  \forall \mathcal{C},
 \; \mathcal{S}(\mathcal{C}[\C_1]) = \mathcal{S}(\mathcal{C}[\C_2])$. \\
 
 Hence, $\mathcal{S}$ is compositional with respect to itself.

\item We note that $\mathcal{S}$ is clearly fully abstract with respect to
itself. This follows from the definition of full abstraction.\\

	By the compositionality property of $[[\cdot]]$,\\
	
	$\forall \mathcal{C}$, $[[\mathcal{C}[\C_1]]] = [[\mathcal{C}[\C_2]]]
	\rightarrow
	\mathcal{S}(\C_1) = \mathcal{S}(\C_2) $.\\

	Therefore, $S$ is fully abstract with respect to $[[\cdot]]$.

\end{itemize}  
  
\end{enumerate}


%% ==================================================================
%% IMPERATIVE REASONING
%% ==================================================================

\section{Imperative reasoning}
\begin{enumerate}

\question{
\item[2.1] Prove or disprove the following statements.
\begin{enumerate}
\item[a)] The programs $(\whilesym\; \B \; \dosym \; \C); (\ifsym \; \B \; \then \; \C_1 \; \elsesym \; \C_2)$ and $(\whilesym \; \B \; \dosym \; \C);\C_2$
are equivalent with respect to the state transition semantics $[[\cdot]]$.

\item[b)] The programs $\whilesym\; \B \; \dosym \; \C$ and $\whilesym \; \B \; \dosym \; (\C;\ifsym \; \B \; \then \; \C \; \elsesym \; \skipsym)$ are 
equivalent with respect to the state transition semantics $[[\cdot]]$.

\item[c)] The programs $\whilesym\; \B \; \dosym \; (\whilesym \; \B \; \dosym \; \C)$ and $\whilesym\; \B \; \dosym \; \C$ are 
contextually equivalent with respect to the state transition semantics $[[\cdot]]$.
\end{enumerate}
}

In this problem, we use the definitions of the state transition semantics
$[[\cdot]]$ :
\begin{align*}
& \mathcal{S}( \; skip \; ) \; = \; \{ (\sigma, \sigma) \; | \; \sigma \in St \}
\\
& \mathcal{S}( \; x := e \; ) \; = \; \{ (\sigma, [\sigma | x : v]) \; | \;
(\sigma, v) \in | e | \; \& \; x \in dom(\sigma) \} \\
& \mathcal{S}( \; \C_1 ; \C_2 ) \;  = \; \{ (\sigma, \sigma'') \; | \; \exists
\; \sigma'.(\sigma, \sigma') \in [[\C_1]] \; \&  \; (\sigma',\sigma'') \in [[\C_2]]
\} \; = \; [[\C_2]] \circ [[\C_1]] \\
& \mathcal{S}( \; \ifsym \; \B \; \then  \; \C_1 \; \elsesym \; \C_2 \; ) \; =
\; \; [[\C_1]] \; \circ  \; [[\B]]_{\true} \; \cup \; [[\C_2]] \; \circ \;
[[\B]]_{\false} \\
& \mathcal{S}( \; \whilesym \; \B \; \dosym \; \C \; ) \; = \; [[\B]]_{\false}
\; \circ \; ([[\C]] \; \circ \; [[\B]]_{\true} )^{*} \\
\end{align*}

\begin{enumerate}

\item[a)] 
\begin{align*}
& \mathcal{S}(\whilesym \; \B \; \dosym \; \C);\C_2) \\
& = [[\C_2]] \circ [[\B]]_{\false} \circ ([[\C]] \circ [[\B]]_{true} )^{*}\\
& \\
& \mathcal{S}(\whilesym\; \B \; \dosym \; \C); (\ifsym \; \B \; \then
\; \C_1 \; \elsesym \; \C_2)) \\
& =  ([[\C_1]] \circ [[\B]]_{\true} \cup [[\C_2]] \circ [[\B]]_{\false})
\circ  
([[\B]]_{\false} \circ ([[\C]] \circ [[\B]]_{\true} )^{*}) \\
& =  [[\C_2]] \circ [[\B]]_{\false} \circ ([[\C]] \circ [[\B]]_{\true} )^{*}
\end{align*}\\

This is because the $\B$ must be $\false$ at the exit of the $\whilesym$ loop
and thus $\C_2$ alone can be executed.
Hence, the two programs are equivalent with respect to $[[\cdot]]$.\\

\item[b)]

\begin{align*}
& \mathcal{S}(\whilesym\; \B \; \dosym \; \C) \\
& = [[\B]]_{\false} \circ ([[\C]] \circ [[\B]]_{\true} )^{*} \\
& \\
& \mathcal{S}(\whilesym \; \B \; \dosym \; (\C;\ifsym \; \B \; \then \; \C \;
\elsesym \; \skipsym)) \\
& = [[\B]]_{\false} \circ ([[\C; \ifsym \; \B \; \then \; \C \;
\elsesym \; \skipsym]] \circ [[b]]_{\true} )^{*} \\
& = [[\B]]_{\false} \circ ( ([[\ifsym \; \B \; \then \; \C \;
\elsesym \; \skipsym]] \circ [[\C]]) \circ [[\B]]_{\true} )^{*} \\
& = [[\B]]_{\false} \circ ( ( ([[\C]] \circ [[\B]]_{\true} \cup [[\skipsym]]
\circ [[\B]]_{\false}) \circ [[\C]]) \circ [[\B]]_{\true} )^{*} \\
& = [[\B]]_{\false} \circ ( ([[\C]] \circ [[\B]]_{\true} 
\circ [[\C]] \circ [[\B]]_{\true} )^{*} \cup 
([[\skipsym]] \circ [[\B]]_{\false} \circ [[\C]] \circ [[\B]]_{\true} )^{*} ) \\
& =  [[\B]]_{\false} \circ ([[\C]] \circ [[\B]]_{\true} )^{*} \\
\end{align*}

This is because $\B$ must be $\true$ within the $\whilesym$ loop
for $\C$ to be executed and this is equivalent to the earlier $\whilesym$ loop.
Hence, the two programs are equivalent with respect to $[[\cdot]]$.\\

\item[c)] 

\begin{align*}
& \mathcal{S}(\whilesym\; \B \; \dosym \; \C) \\
& = [[\B]]_{\false} \circ ([[\C]] \circ | b |_{\true} )^{*} \\
& \\
& \mathcal{S}(\whilesym\; \B \; \dosym \; (\whilesym \; \B \; \dosym \; \C)\\
& = [[\B]]_{\false} \circ ([[ \; [[\B]]_{\false} \circ ([[\C]] \circ
[[\B]]_{\true} )^{*} ]] \circ | b |_{\true} )^{*} \\
& = [[\B]]_{\false} \circ ([[\C]] \circ [[\B]]_{\true} \circ [[\B]]_{\true}
)^{*}
\\
& = [[\B]]_{\false} \circ ([[\C]] \circ [[\B]]_{\true} )^{*} \\
\end{align*}

This is because $\B$ must be $\true$ within the inner $\whilesym$ loop
for $\C$ to be executed and this is equivalent to the earlier $\whilesym$ loop.
Hence, the two programs are equivalent with respect to $[[\cdot]]$.\\

As $[[\cdot]]$ is compositional, thus, \\

$[[\C_1]] = [[\C_2]] \rightarrow \forall \mathcal{C}, \;	
	[[\mathcal{C}[\C_1]]] = [[\mathcal{C}[\C_2]]]$.\\

Hence, the two programs are contextually equivalent with respect to $[[\cdot]]$.

\end{enumerate}


\question{
\item[2.2] Under what conditions are the programs
\begin{enumerate} 

\item[a)] $\C;(\ifsym \; \B \; \then\; \C_1\; \elsesym \; \C_2)$ and $\ifsym \; \B \; \then\; (\C; \C_1) \; \elsesym \; (\C; \C_2)$
\item[b)] $(\ifsym \; \B \; \then\; \C_1 \; \elsesym \; \C_2); (\ifsym \; \B \; \then \; \C_3 \; \elsesym \; \C_4)$ and $\ifsym \; \B \; \then\; (\C_1;\C_3) \; \elsesym \; (\C_2;\C_4)$
\end{enumerate}

equivalent with respect to the state transition semantics $[[\cdot]]$?
}

\begin{enumerate}
  
\item[a)] 

$P_1 = (\C;(\ifsym \; \B_1 \; \then\; \C_1\; \elsesym \; \C_2)$, and\\
$P_2 = \ifsym \; \B_2 \; \then\; (\C; \C_1) \; \elsesym \; (\C; \C_2)$.\\

Intuitively, executing $\C$ must not alter the initial truth value of $\B$.
Otherwise, we execute along one path in $P_1$, and another path in $P_2$. \\

Formally, we require that $[[\B_1]] = [[\B_2]]$. 
This can happen in two ways. 
$( (free(\B_1) \cap writes(\C) = \phi) \; \orsym \; ( free(\B_1) \cap
writes(\C) \neq \phi \; \andsym \; [[\B_1]] = [[\B_2]]) )$.\\

This ensures that $[[P_1]] = [[P_2]]$. \\

\item[b)] 

$P_1 = (\ifsym \; \B_1 \; \then\; \C_1 \; \elsesym \; \C_2); (\ifsym \; \B_2 \;
\then \; \C_3 \; \elsesym \; \C_4)$, and\\
$P_2 = \ifsym \; \B_3 \; \then\; (\C_1;\C_3) \; \elsesym \; (\C_2;\C_4)$.\\

Intuitively, if $\B_1$ is true, we require that
executing $\C_1$ must not alter the truth value of $\B_2$, and, 
if $\B_1$ is false, we require that executing 
$\C_2$ must not alter the truth value of $\B_2$.
Otherwise, we execute along one path in $P_1$ and another path in $P_2$. \\

Formally, we require that $[[\B_1]] = [[\B_2]]$. This can happen in two ways.
Either, $\B_1 = \true$ $\andsym$ 
$( (free(\B_2) \cap writes(\C_1) = \phi) \; \orsym \; ( free(\B_2) \cap
writes(\C_1) \neq \phi \; \andsym \; [[\B_1]] = [[\B_2]]) )$.\\
Or, $\B_1 = \false$ $\andsym$ 
$( (free(\B_2) \cap writes(\C_2) = \phi) \; \orsym \; ( free(\B_2) \cap
writes(\C_2) \neq \phi \; \andsym \; [[\B_1]] = [[\B_2]]) )$.\\

This ensures that $[[P_1]] = [[P_2]]$.\\

\end{enumerate}


\question{
\item[2.3] Show that the following program computes the greatest common divisor of two positive integers $x,y$:

\begin{align*}
& \whilesym \; \notsym(x = y) \; \dosym \\
& \;\;\;\;\; \whilesym \; (x < y) \; \dosym \; y := y - x; \\
& \;\;\;\;\; \whilesym \; (y < x) \; \dosym \; x := x - y
\end{align*}

Here we write $\E_1 < \E_2$ for $\E_1 \leq \E_2 \; \andsym \; \notsym(\E_1 = \E_2)$ and $\E_1 - \E_2$ for $\E_1 + \negation(\E_2)$. 
What are the values of the above program under the input/output and termination semantics? 
How do your answers change if we instead consider the program

\begin{align*}
& \whilesym \; \notsym(x = y) \; \dosym \\
& \;\;\;\;\; \whilesym \; {\color{red} (x \leq y)} \; \dosym \; y := y - x; \\
& \;\;\;\;\; \whilesym \; {\color{red} (y \leq x)} \; \dosym \; x := x - y
\end{align*}
}

\end{enumerate}

\end{document}


