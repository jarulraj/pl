
\documentclass{article}

\usepackage[usenames,dvipsnames]{xcolor}
\usepackage{amsmath}
\usepackage{latexsym,amsthm,amssymb,amscd,url,enumerate}
\usepackage[show]{ed}
\usepackage[all]{xy}
\usepackage{tikz}
\usepackage{tikz-cd}
\usepackage{leftidx}
\usetikzlibrary{arrows}
\usepackage{amsmath}
\usepackage{amssymb}
\usepackage{amsthm}
\usepackage{stmaryrd}
\usepackage{wasysym}
\usepackage{proof}
\usepackage{hyperref}
\usepackage{framed}
\usepackage{fullpage}

\newtheorem{theorem}{Theorem}						
\newtheorem{definition}[theorem]{Definition}
\newtheorem{proposition}[theorem]{Proposition}
\newtheorem{lemma}[theorem]{Lemma}
\newtheorem{axiom}{Axiom}
\newtheorem{notation}[theorem]{Notation}
\newtheorem{corollary}[theorem]{Corollary}

%% ==================================================================
%% MACROS
%% ==================================================================

\newcommand{\E}{\mathtt{E}}
\newcommand{\B}{\mathtt{B}}
\newcommand{\C}{\mathtt{C}}
\newcommand{\R}{\mathtt{R}}
\newcommand{\N}{\mathtt{N}}
\newcommand{\LL}{\mathtt{L}}
\newcommand{\true}{\mathtt{true}}
\newcommand{\false}{\mathtt{false}}
\newcommand{\andsym}{\mathtt{and}}
\newcommand{\orsym}{\mathtt{or}}
\newcommand{\notsym}{\mathop{\mathtt{not}}}
\newcommand{\ifsym}{\mathtt{if}}
\newcommand{\then}{\mathtt{then}}
\newcommand{\elsesym}{\mathtt{else}}
\newcommand{\whilesym}{\mathtt{while}}
\newcommand{\dosym}{\mathtt{do}}
\newcommand{\skipsym}{\mathtt{skip}}
\newcommand{\nil}{\mathtt{nil}}
\newcommand{\case}{\mathtt{case}}
\newcommand{\digit}{\mathtt{d}}
\newcommand{\negation}{\mathtt{neg}}
\newcommand{\Digit}{\mathbf{Digit}}
\newcommand{\denot}[1]{\mathtt{[[{#1}]]}}
\newcommand{\Sem}{\mathtt{S}}

\newcommand{\G}{\Gamma}
\newcommand{\D}{\Delta}

\newcommand{\type}{\;\mathsf{type}}
\newcommand{\val}{\;\mathsf{val}}
\newcommand{\dom}[1]{\mathsf{dom}(#1)}
\newcommand{\FV}{\mathsf{FV}}
\newcommand{\reduces}{\mapsto}

\newcommand{\ctx}[3]{(#1,#2\rhd #3)}
\newcommand{\bool}{\mathbf{2}}
\newcommand{\boolt}{\mathsf{tt}}
\newcommand{\boolf}{\mathsf{ff}}
\newcommand{\sem}[1]{\llbracket #1 \rrbracket}

% Logical equivalence
\newcommand{\rels}[3]{#1:#2\leftrightarrow #3}
\newcommand{\relto}{\hookrightarrow}
\newcommand{\LR}[2]{\sem{#1}_{#2}}

\newcommand{\gd}[1]{\hat\gamma(\hat\delta(#1))}
\newcommand{\gdp}[1]{\widehat{\gamma'}(\widehat{\delta'}(#1))}

% Existential types
\newcommand{\pack}[3]{\mathsf{pack}\,#1\,\mathsf{with}\,#2\,\mathsf{as}\,#3}
\newcommand{\open}[4]{\mathsf{open}\,#1\,\mathsf{as}\,#2\,\mathsf{with}\,#3\,\mathsf{in}\,#4}

% Data types
\newcommand{\Nat}{\mathsf{nat}}
\newcommand{\z}{\mathsf{z}}
\newcommand{\suc}{\mathsf{suc}}
\newcommand{\natrec}{\mathsf{natrec}}
\newcommand{\pred}{\mathsf{pred}}
\newcommand{\sub}{\mathsf{sub}}

\newcommand{\List}{\mathsf{list}}
\newcommand{\cons}{\mathsf{cons}}
\newcommand{\foldr}{\mathsf{foldr}}

\newcommand{\fst}{\mathsf{fst}}
\newcommand{\snd}{\mathsf{snd}}

\newcommand{\oList}{\overline{\mathsf{list}}}
\newcommand{\onil}{\overline{\mathsf{nil}}}
\newcommand{\ocons}{\overline{\mathsf{cons}}}
\newcommand{\ohead}{\overline{\mathsf{head}}}
\newcommand{\otail}{\overline{\mathsf{tail}}}
\newcommand{\ofoldr}{\overline{\mathsf{foldr}}}

% crazy hacks
\makeatletter
\def\lam#1{{\lambda}\@lamarg#1:\@endlamarg\@ifnextchar\bgroup{.\,\lam}{.\,}}
\def\@lamarg#1:#2\@endlamarg{\if\relax\detokenize{#2}\relax #1\else\@lamvar{\@lameatcolon#2},#1\@endlamvar\fi}
\def\@lamvar#1,#2\@endlamvar{#2\,{:}\,#1}
\def\@lameatcolon#1:{#1}
\let\lamt\lam
\makeatother

% disable hbox warnings
\hfuzz=5.002pt 

%% ==================================================================
%% HELPERS
%% ==================================================================

\definecolor{DarkBlue}{rgb}{0.00,0.30,0.90}

\newcommand{\question}[1]
{\color{DarkBlue}#1 \color{Black}}

\definecolor{comment-color}{rgb}{1,0,0}
\renewcommand{\todo}[1]{
\textnormal{\color{comment-color}{\textbf{TODO: #1}}}\unskip}

%% ==================================================================
%% DOCUMENT
%% ==================================================================

\begin{document}

\title{\textbf{15-812 Semantics of Programming Languages \\ Spring 2015 \\
{\large Assignment 2 Solutions}}\\\vspace{0.3in}
{\Large \bf Joy Arulraj (AID : jarulraj)}}

\date{}
\author{}

\maketitle

\question{
\noindent In this assignment we will revisit the imperative language $\mathcal{L}$ defined by the syntax
\begin{align*}
& \E := \mathtt{n} \; | \; \mathtt{x} \; | \; \E_1 + \E_2 \; | \; \E_1 \cdot \E_2 \; | \; \negation(\E) \; | \; \ifsym \; \B \; \then \; \E_1 \; \elsesym \; \E_2 \\ 
& \B := \true \; | \; \false \; | \; \B_1 \; \andsym \; \B_2 \; | \; \B_1 \; \orsym \; \B_2 \; | \; \notsym \; \B \; | \; \E_1 \leq \E_2 \\
& \C := \skipsym \; | \; \mathtt{x} := \E \; | \; \C_1 ; \C_2 \; | \; \ifsym \; \B \; \then \; \C_1 \; \elsesym \; \C_2 \; | \; \whilesym \; \B \; \dosym \; \C
\end{align*}
and the state-transition $[[\cdot]]$, input/output $\mathcal{I}$, and termination $\mathcal{T}$ semantics for $\mathcal{L}$ as defined in class. Recall that two programs $\C_1, \C_2$ are 
\begin{enumerate}
\item \emph{equivalent} with respect to a semantics $\Sem$ if $\Sem(\C_1) = \Sem(\C_2)$
\item \emph{contextually equivalent} with respect to $\Sem$ if $\forall \mathcal{C}, \Sem(\mathcal{C}[\C_1]) = \Sem(\mathcal{C}[\C_2])$, where $\mathcal{C}$ ranges over program contexts
\end{enumerate}
For two semantics $\Sem_1, \Sem_2$ we say that
\begin{enumerate}
	\item $\Sem_1$ is \emph{compositional} with respect to $\Sem_2$ if for any two programs $\C_1,\C_2$, we have that $\Sem_1(\C_1) = \Sem_1(\C_2)$ implies $\forall \mathcal{C}, \Sem_2(\mathcal{C}[\C_1]) = \Sem_2(\mathcal{C}[\C_2])$
	\item $\Sem_1$ is \emph{fully abstract} with respect to $\Sem_2$ if for any two programs $\C_1,\C_2$, we have that $\forall \mathcal{C}, \Sem_2(\mathcal{C}[\C_1]) = \Sem_2(\mathcal{C}[\C_2])$ implies $\Sem_1(\C_1) = \Sem_1(\C_2)$
	\end{enumerate}
	A semantics $\Sem$ is \emph{compositional} if it is compositional with respect to itself.\bigskip

\noindent \textbf{Please justify all answers.}
}

%% ==================================================================
%% REVISITING SEMANTICS
%% ==================================================================

\section{Revisiting semantics}

\begin{enumerate}
\question{
\item[1.1] 
\begin{enumerate}
	\item[a)] Is the input/output semantics $\mathcal{I}$ compositional? Is it
	fully abstract with respect to the termination semantics $\mathcal{T}$?
	\item[b)] Is the termination semantics $\mathcal{T}$ compositional? Is it
	fully abstract with respect to the input/output semantics $\mathcal{I}$?
\end{enumerate}
}

\question{	
\item[1.2]
\begin{enumerate}
	\item[a)] Give a semantics $\Sem$ such that the state transition semantics 
	$[[\cdot]]$ is not compositional with respect to $\Sem$.
	\item[b)] Give a semantics $\Sem$ such that the state transition semantics 
	$[[\cdot]]$ is not fully abstract with respect to $\Sem$.
\end{enumerate}
}


\question{
\item[1.3] In the previous homework assignment, question 2, you defined a denotational 
semantics $\Sem$ for the extended imperative language $\mathcal{L}_e$:

\begin{align*}
& \E := \mathtt{n} \; | \; \mathtt{x} \; | \; \E_1 + \E_2 \; | \; \E_1 \cdot \E_2 \; | \; \negation(\E) \; | \; \ifsym \; \B \; \then \; \E_1 \; \elsesym \; \E_2 \\ 
& \B := \true \; | \; \false \; | \; \B_1 \; \andsym \; \B_2 \; | \; \B_1 \; \orsym \; \B_2 \; | \; \notsym \; \B \; | \; \E_1 \leq \E_2 \\
& {\color{red} \LL := \nil \; | \; (\mathtt{n},\C)::\LL} \\
& \C := \skipsym \; | \; \mathtt{x} := \E \; | \; \C_1 ; \C_2 \; | \; \ifsym \; \B \; \then \; \C_1 \; \elsesym \; \C_2 \; | \; \whilesym \; \B \; \dosym \; \C \; | \; {\color{red} \case \; \E \; \LL \; \C}
\end{align*}

Is $\Sem$ compositional? Is the state transition semantics $[[\cdot]]$ compositional with respect to $\Sem$? 
Is it fully abstract?
}

\end{enumerate}


%% ==================================================================
%% IMPERATIVE REASONING
%% ==================================================================

\section{Imperative reasoning}
\begin{enumerate}

\question{
\item[2.1] Prove or disprove the following statements.
\begin{enumerate}
\item[a)] The programs $(\whilesym\; \B \; \dosym \; \C); (\ifsym \; \B \; \then \; \C_1 \; \elsesym \; \C_2)$ and $(\whilesym \; \B \; \dosym \; \C);\C_2$
are equivalent with respect to the state transition semantics $[[\cdot]]$.

\item[b)] The programs $\whilesym\; \B \; \dosym \; \C$ and $\whilesym \; \B \; \dosym \; (\C;\ifsym \; \B \; \then \; \C \; \elsesym \; \skipsym)$ are 
equivalent with respect to the state transition semantics $[[\cdot]]$.

\item[c)] The programs $\whilesym\; \B \; \dosym \; (\whilesym \; \B \; \dosym \; \C)$ and $\whilesym\; \B \; \dosym \; \C$ are 
contextually equivalent with respect to the state transition semantics $[[\cdot]]$.
\end{enumerate}
}

\question{
\item[2.2] Under what conditions are the programs
\begin{enumerate} 

\item[a)] $\C;(\ifsym \; \B \; \then\; \C_1\; \elsesym \; \C_2)$ and $\ifsym \; \B \; \then\; (\C; \C_1) \; \elsesym \; (\C; \C_2)$
\item[b)] $(\ifsym \; \B \; \then\; \C_1 \; \elsesym \; \C_2); (\ifsym \; \B \; \then \; \C_3 \; \elsesym \; \C_4)$ and $\ifsym \; \B \; \then\; (\C_1;\C_3) \; \elsesym \; (\C_2;\C_4)$
\end{enumerate}

equivalent with respect to the state transition semantics $[[\cdot]]$?
}

\question{
\item[2.3] Show that the following program computes the greatest common divisor of two positive integers $x,y$:

\begin{align*}
& \whilesym \; \notsym(x = y) \; \dosym \\
& \;\;\;\;\; \whilesym \; (x < y) \; \dosym \; y := y - x; \\
& \;\;\;\;\; \whilesym \; (y < x) \; \dosym \; x := x - y
\end{align*}

Here we write $\E_1 < \E_2$ for $\E_1 \leq \E_2 \; \andsym \; \notsym(\E_1 = \E_2)$ and $\E_1 - \E_2$ for $\E_1 + \negation(\E_2)$. 
What are the values of the above program under the input/output and termination semantics? 
How do your answers change if we instead consider the program

\begin{align*}
& \whilesym \; \notsym(x = y) \; \dosym \\
& \;\;\;\;\; \whilesym \; {\color{red} (x \leq y)} \; \dosym \; y := y - x; \\
& \;\;\;\;\; \whilesym \; {\color{red} (y \leq x)} \; \dosym \; x := x - y
\end{align*}
}

\end{enumerate}

\end{document}


