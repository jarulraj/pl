
\documentclass[a4paper,10pt]{article}

\usepackage{amsmath}
\usepackage[usenames,dvipsnames]{xcolor}
\usepackage{latexsym,amsthm,amssymb,amscd,url,enumerate}
\usepackage[show]{ed}
\usepackage[all]{xy}
\usepackage{tikz}
\usepackage{tikz-cd}
\usepackage{leftidx}
\usetikzlibrary{arrows}
\usepackage{amsmath}
\usepackage{amssymb}
\usepackage{amsthm}
\usepackage{stmaryrd}
\usepackage{wasysym}
\usepackage{proof}
\usepackage{hyperref}
\usepackage{framed}
\usepackage{booktabs}
\usepackage{fullpage}

%% ==================================================================
%% MACROS
%% ==================================================================

\newcommand{\E}{\mathtt{E}}
\newcommand{\B}{\mathtt{B}}
\newcommand{\C}{\mathtt{C}}
\newcommand{\R}{\mathtt{R}}
\newcommand{\N}{\mathtt{N}}
\newcommand{\LL}{\mathtt{L}}
\newcommand{\true}{\mathtt{true}}
\newcommand{\false}{\mathtt{false}}
\newcommand{\andsym}{\mathtt{and}}
\newcommand{\orsym}{\mathtt{or}}
\newcommand{\notsym}{\mathop{\mathtt{not}}}
\newcommand{\ifsym}{\mathtt{if}}
\newcommand{\then}{\mathtt{then}}
\newcommand{\elsesym}{\mathtt{else}}
\newcommand{\whilesym}{\mathtt{while}}
\newcommand{\dosym}{\mathtt{do}}
\newcommand{\skipsym}{\mathtt{skip}}
\newcommand{\nil}{\mathtt{nil}}
\newcommand{\case}{\mathtt{case}}
\newcommand{\digit}{\mathtt{d}}
\newcommand{\negation}{\mathtt{neg}}
\newcommand{\Digit}{\mathbf{Digit}}
\newcommand{\denot}[1]{\mathtt{[[{#1}]]}}

\newcommand{\G}{\Gamma}
\newcommand{\D}{\Delta}

\newcommand{\type}{\;\mathsf{type}}
\newcommand{\val}{\;\mathsf{val}}
\newcommand{\dom}[1]{\mathsf{dom}(#1)}
\newcommand{\FV}{\mathsf{FV}}
\newcommand{\reduces}{\mapsto}

\newcommand{\ctx}[3]{(#1,#2\rhd #3)}
\newcommand{\bool}{\mathbf{2}}
\newcommand{\boolt}{\mathsf{tt}}
\newcommand{\boolf}{\mathsf{ff}}
\newcommand{\sem}[1]{\llbracket #1 \rrbracket}

% Logical equivalence
\newcommand{\rels}[3]{#1:#2\leftrightarrow #3}
\newcommand{\relto}{\hookrightarrow}
\newcommand{\LR}[2]{\sem{#1}_{#2}}

\newcommand{\gd}[1]{\hat\gamma(\hat\delta(#1))}
\newcommand{\gdp}[1]{\widehat{\gamma'}(\widehat{\delta'}(#1))}

% Existential types
\newcommand{\pack}[3]{\mathsf{pack}\,#1\,\mathsf{with}\,#2\,\mathsf{as}\,#3}
\newcommand{\open}[4]{\mathsf{open}\,#1\,\mathsf{as}\,#2\,\mathsf{with}\,#3\,\mathsf{in}\,#4}

% Data types
\newcommand{\Nat}{\mathsf{nat}}
\newcommand{\z}{\mathsf{z}}
\newcommand{\suc}{\mathsf{suc}}
\newcommand{\natrec}{\mathsf{natrec}}
\newcommand{\pred}{\mathsf{pred}}
\newcommand{\sub}{\mathsf{sub}}

\newcommand{\List}{\mathsf{list}}
\newcommand{\cons}{\mathsf{cons}}
\newcommand{\foldr}{\mathsf{foldr}}

\newcommand{\fst}{\mathsf{fst}}
\newcommand{\snd}{\mathsf{snd}}

\newcommand{\oList}{\overline{\mathsf{list}}}
\newcommand{\onil}{\overline{\mathsf{nil}}}
\newcommand{\ocons}{\overline{\mathsf{cons}}}
\newcommand{\ohead}{\overline{\mathsf{head}}}
\newcommand{\otail}{\overline{\mathsf{tail}}}
\newcommand{\ofoldr}{\overline{\mathsf{foldr}}}

% crazy hacks
\makeatletter
\def\lam#1{{\lambda}\@lamarg#1:\@endlamarg\@ifnextchar\bgroup{.\,\lam}{.\,}}
\def\@lamarg#1:#2\@endlamarg{\if\relax\detokenize{#2}\relax #1\else\@lamvar{\@lameatcolon#2},#1\@endlamvar\fi}
\def\@lamvar#1,#2\@endlamvar{#2\,{:}\,#1}
\def\@lameatcolon#1:{#1}
\let\lamt\lam
\makeatother

\newtheorem{theorem}{Theorem}						
\newtheorem{definition}[theorem]{Definition}
\newtheorem{proposition}[theorem]{Proposition}
\newtheorem{lemma}[theorem]{Lemma}
\newtheorem{axiom}{Axiom}
\newtheorem{notation}[theorem]{Notation}
\newtheorem{corollary}[theorem]{Corollary}

% disable hbox warnings
\hfuzz=5.002pt 

%% ==================================================================
%% HELPERS
%% ==================================================================

\definecolor{DarkBlue}{rgb}{0.00,0.30,0.90}

\newcommand{\question}[1]
{\color{DarkBlue}#1 \color{Black} \newline}

\definecolor{comment-color}{rgb}{1,0,0}
\renewcommand{\todo}[1]{
\textnormal{\color{comment-color}{\textbf{TODO: #1}}}\unskip}

%% ==================================================================
%% DOCUMENT
%% ==================================================================

\begin{document}

\title{\textbf{15-812 Semantics of Programming Languages \\ Spring 2015 \\
{\large Assignment 1 Solutions}}\\\vspace{0.3in}
{\Large \bf Joy Arulraj (AID : jarulraj)}}

\date{}
\author{}

\maketitle

%% ==================================================================
%% BALANCED TERNARY EXPRESSIONS
%% ==================================================================

\section{Balanced ternary expressions}

\question{
This homework problem concerns a non-standard arithmetical language based on balanced 
ternary notation (using digits with values -1, 0, 1). This notation was actually used in some early compilers,
and has some technical advantages over conventional ternary (which uses 
digits with values 0, 1, 2). In particular, one can represent positive and negative 
integers with equal ease, symbolic arithmetic is straightforward, and negation 
corresponds to a very simple symbolic operation.

Consider the following abstract syntax for balanced ternary expressions $\E$ 
where $\digit$ ranges over the set $\Digit = \{\mathtt{+},\mathtt{0},\mathtt{-}\}$:

\begin{align*}
\E := \digit \; | \; \E \digit \; | \; \E_1 \oplus \E_2 \; | \; \E_1 \star \E_2 \; | \; \negation(\E)
\end{align*}
}
\begin{enumerate}

\question{
\item[1.1] Define formally the notion of a reduced ternary numeral (either using abstract syntax or inference rules).
}

A ternary numeral without redundant leading zeros is defined as a reduced
ternary numeral. Formally, 

\begin{gather*}
R := d \; | \; Rd \; (R \neq \mathtt{0}) \\
d := + \; | \mathtt{0} \; | \; - 
\end{gather*}

\question{
\item[1.2] Define an appropriate denotational semantics $\denot{\cdot}$ into the
set $\mathbb{Z}$ of integers, with $\oplus, \; \star, \; \negation$ representing 
the obvious arithmetic functions of addition, multiplication, and negation.
}

Let \textbf{Exp} be the set of balanced ternary expressions. Then, we define
$$\denot{\cdot} : \textbf{Exp} \rightarrow \mathbb{Z}$$ thus,

\begin{gather*}
\denot{0} = 0 \\
\denot{+} = +1 \\
\denot{-} = -1 \\
\end{gather*}
\begin{gather*}
\denot{E0} = 3*\denot{E} \\
\denot{E+} = 3*\denot{E} + 1 \\
\denot{E-} = 3*\denot{E} - 1 \\
\end{gather*}
\begin{gather*}
\denot{E_{1} \oplus E_{2}} = \denot{E_{1}} \oplus \denot{E_{2}} \\
\denot{E_{1} \star E_{2}} = \denot{E_{1}} \star \denot{E_{2}} \\
\denot{neg(E)} = -\denot{E} \\
\end{gather*}

\question{
\item[1.3] Find a reduced numeral $\R$ for which $\denot{\R} = 42$ and
explain why it has the required property. Is such a reduced numeral unique? If
so, why?
}

$$ \R = +---0 $$  Because, $$\denot{\R} = (1) * 3^{4} - (1) * 3^{3} + (1) *
3^{2} + (1) * 3^{1} = 42$$

Yes, each reduced numeral $\R$ maps to an unique integer. We show this by
induction on the depth of $\R$.

\proof
If depth is 0, by definition, $\denot{0} = 0$, $\denot{+} = +1$ and
$\denot{-} = -1$. Thus, each numeral $\R$ maps to an unique integer i.e.
for each of these 3 integers, there exists only one such numeral such that
$\denot{\R}$ evaluates to that integer.

Assuming that numerals of depth $k$ map to unique integers, we show
that numeral of depth $k+1$ also map to unique integers.
For a numeral's depth to decrease by 1, the only valid rule is
to transition from $\R d$ to $\R$. Now, $\denot{\R d} = 3 * \denot{\R} + d$.
As the set of integers that correspond to numerals of depth $k$ by the
induction hypothesis, 
each numeral $\R$ of depth $k+1$ also maps to an unique integer.

\question{
\item[1.4] Define a ternary sum operator $+_3 : \Digit \times \Digit \to \Digit \times \Digit$ 
such that if $d_1 +_3 d_2 = (d_3,d_4)$, then $\denot{\digit_1} + \denot{\digit_2} = 3
\denot{\digit_3} + \denot{\digit_4}$ for any digits $\digit_1, \digit_2, \digit_3, \digit_4$.
}

\begin{table}[h!]
    \centering
	
	$$ +_3 : \Digit \times \Digit \to \Digit \times \Digit$$ 

	\begin{tabular}{l | c c c}
	$+_3$ & - & 0 & + \\
	\hline
	- & (-, +) & (0, -) & (0, 0) \\
	0 & (0, -) & (0, 0) & (0, +) \\
	+ & (0, 0) & (0, +) & (+, -) \\
	\end{tabular}
\end{table}

\question{
\item[1.5] Define an appropriate operational semantics for ternary expressions, 
with $\E \to \E'$ denoting the reduction relation.
}

\[
\infer[d \in \Digit]
  {0d \rightarrow d}
  {}   
\qquad
\infer[] 
  {neg(0) \rightarrow 0}  
  {}    
\qquad
\infer[] 
  {neg(+) \rightarrow -}  
  {}    
\qquad
\infer[] 
  {neg(-) \rightarrow +}  
  {}    
\]

\hrule

\[
\infer[d \in \Digit]
  {Ed \rightarrow E'd}
  {E \rightarrow E'}
\]

\[
\infer[]
  {E_{1} \oplus E_{2} \rightarrow E_{1}' \oplus E_{2}}
  {E_{1} \rightarrow E_{1}'} 
\qquad
\infer[]
  {R \oplus E_{2} \rightarrow R \oplus E_{2}'}
  {E_{2} \rightarrow E_{2}'}
\]

\[
\infer[]
  {E_{1} \star E_{2} \rightarrow E_{1}' \star E_{2}}
  {E_{1} \rightarrow E_{1}'} 
\qquad
\infer[d \in \Digit]
  {R \star E_{2} \rightarrow R \star E_{2}'}
  {E_{2} \rightarrow E_{2}'}
\]
\[
\infer[]
  {neg(E) \rightarrow neg(E')}
  {E \rightarrow E'}
\]

\hrule

\[
\infer[(c, d) = d_{1} +_3 d_{2}] 
  {d_{1} \oplus d_{2} \rightarrow cd}
  {}  
\]
\[
\infer[(c, d) = d_{1} +_3 d_{2}] 
  {d_{1} \oplus (R_{2} d_{2}) \rightarrow (c \oplus R_{2}) d}
  {}  
\]
\[
\infer[(c, d) = d_{1} +_3 d_{2}] 
  {(R_{1} d_{1}) \oplus d_{2}  \rightarrow (R_{1} \oplus c) d}
  {}  
\]
\[
\infer[(c, d) = d_{1} +_3 d_{2}] 
  {(R_{1} d_{1}) \oplus (R_{2} d_{2}) \rightarrow ((R_{1} + R_{2}) \oplus c) d}
  {}  
\]

\hrule

\[
\infer[(c, d) = d_{1} \star_{3} d_{2}] 
  {d_{1} \star d_{2} \rightarrow cd}
  {}  
\]
\[
\infer[(c, d) = d_{1} \star_{3} d_{2}] 
  {d_{1} \star (R_{2} d_{2}) \rightarrow (c \oplus R_{2}) d}
  {}  
\]
\[
\infer[(c, d) = d_{1} \star_{3} d_{2}] 
  {(R_{1} d_{1}) \star d_{2}  \rightarrow (R_{1} \oplus c) d}
  {}  
\]
\[
\infer[(c, d) = d_{1} \star_{3} d_{2}] 
  {(R_{1} d_{1}) \star (R_{2} d_{2}) \rightarrow ((R_{1} + R_{2}) \star c) d}
  {}  
\]

\hrule

\[
\infer[d' = neg(d)] 
  {neg(Ed) \rightarrow neg(E')d'}  
  {E \rightarrow E'}    
\]

\question{
\item[1.6] Show that for every expression $\E$, $\E$ has no transitions if and 
only if it is a reduced numeral.
}

\begin{itemize}

	\item {\textbf{If $\E$ is a reduced numeral, then it has no transitions.}}
	
	By definition of reduced numeral, it has can be only of the form :
	
\begin{gather*}
R := d \; | \; Rd \; (R \neq \mathtt{0}) \\
d := + \; | \mathtt{0} \; | \; - 
\end{gather*}
	
	Note that it has no leading zeroes. We observe that none of the preconditions
	for the transition rules defined in 1.5 hold. Thus, 
	it has no possible transitions.
	
	\item {\textbf{If $\E$ has no transitions, then it is a reduced numeral.}}
	
	Let's prove this by contradiction. Assume that $\E$ is a reduced numeral and it
	has a valid transition.
	Then it can either transition to a reduced numeral or to an expression 
	that is not a reduced numeral. By the definition of	the reduced numeral, 
	$\E$ is not a reduced numeral. Hence, this is a contradiction.

\end{itemize}

\question{
\item[1.7] Show that every transition $\E \to \E'$ preserves denotational semantics.
}

\[
\infer[d \in \Digit]
  {Ed \rightarrow E'd}
  {E \rightarrow E'}
\]

\proof
\begin{eqnarray*}
\denot{Ed}  
            & = &  3 * \denot{E} + d  \\
            & = &  3 * \denot{E'} + d \qquad \text{(By ref. trans.)} \\ 
            & = & \denot{E'd} 
\end{eqnarray*}

\[
\infer[]
  {E_{1} \oplus E_{2} \rightarrow E_{1}' \oplus E_{2}}
  {E_{1} \rightarrow E_{1}'} \qquad
\infer[]
  {R \oplus E_{2} \rightarrow R \oplus E_{2}'}
  {E_{2} \rightarrow E_{2}'}
\]

\proof:
\begin{eqnarray*}
\denot{E_{1} \oplus E_{2}}  
            & = &  \denot{E_{1}} \oplus \denot{E_{2}} \\
            & = &  \denot{E_{1}'} \oplus \denot{E_{2}}  
            \qquad \text{(By ref. trans.)}
            \\ 
            & = & \denot{E_{1}' \oplus E_{2}} 
\end{eqnarray*}

\begin{eqnarray*}
\denot{R \oplus E_{2}}  
            & = &  \denot{R} \oplus \denot{E_{2}} \\
            & = &  \denot{R} \oplus \denot{E_{2}'}  
            \qquad \text{(By ref. trans.)} \\
            & = & \denot{{R} \oplus E_{2}'} 
\end{eqnarray*}

\[
\infer[]
  {E_{1} \star E_{2} \rightarrow E_{1}' \star E_{2}}
  {E_{1} \rightarrow E_{1}'} \qquad
\infer[d \in \Digit]
  {R \star E_{2} \rightarrow R \star E_{2}'}
  {E_{2} \rightarrow E_{2}'}
\]

\proof:

\begin{eqnarray*}
\denot{E_{1} \star E_{2}}  
            & = &  \denot{E_{1}} \star \denot{E_{2}} \\
            & = &  \denot{E_{1}'} \star \denot{E_{2}} \qquad \text{(By ref. trans.)}\\  
            & = & \denot{E_{1}' \star E_{2}} 
\end{eqnarray*}


\begin{eqnarray*}
\denot{R \star E_{2}}  
            & = &  \denot{R} \star \denot{E_{2}} \\
            & = &  \denot{R} \star \denot{E_{2}'} \qquad \text{(By ref. trans.)}\\  
            & = & \denot{{R} \star E_{2}'} 
\end{eqnarray*}

\[
\infer[]
  {neg(E) \rightarrow neg(E')}
  {E \rightarrow E'}
\]

\proof:
\begin{eqnarray*}
\denot{neg(E)}  
            & = &  -\denot{E} \\
            & = &  -\denot{E'}  \qquad \text{(By ref. trans.)} \\
            & = & \denot{neg(E')} 
\end{eqnarray*}

\hrule

\[
\infer[(c, d) = d_{1} +_3 d_{2}] 
  {d_{1} \oplus d_{2} \rightarrow cd}
  {}  
\]

\proof:
\begin{eqnarray*}
\denot{d_{1} \oplus d_{2}}  
            & = & \denot{d_{1}} + \denot{d_{2}} \\
            & = & 3 * c + d  \\            
            & = & \denot{cd} 
\end{eqnarray*}

\[
\infer[(c, d) = d_{1} +_3 d_{2}] 
  {d_{1} \oplus (R_{2} d_{2}) \rightarrow (c \oplus R_{2}) d}
  {}  
\]

\proof:
\begin{eqnarray*}
\denot{d_{1} \oplus (R_{2} d_{2})}  \\
            & = & \denot{d_{1}} + \denot{R_{2} d_{2}} \\
            & = & d_{1} + 3 * R_{2} + d_{2} \\
            & = & 3 * R_{2} +  3 * c + d \\
            & = & 3 * (c + R_{2}) + d \\
            & = & \denot{(c \oplus R_{2}) d}       
\end{eqnarray*}

\[
\infer[(c, d) = d_{1} +_3 d_{2}] 
  {(R_{1} d_{1}) \oplus d_{2}  \rightarrow (R_{1} \oplus c) d} 
  {}  
\]

\proof:
\begin{eqnarray*}
\denot{(R_{1} d_{1}) \oplus R_{2}}  \\
            & = & \denot{R_{1} d_{1}} + \denot{d_{2}} \\
            & = & 3 * R_{1} + d_{1} + d_{2} \\
            & = & 3 * R_{1} +  3 * c + d \\
            & = & 3 * (R_{1} + c) + d \\
            & = & \denot{(R_{1} \oplus c) d}    
\end{eqnarray*}

\[
\infer[(c, d) = d_{1} +_3 d_{2}] 
  {(R_{1} d_{1}) \oplus (R_{2} d_{2}) \rightarrow ((R_{1} + R_{2}) \oplus c) d}
  {}  
\]

\proof:
\begin{eqnarray*}
\denot{(R_{1} d_{1}) \oplus (R_{2} d_{2})}  
            & = & \denot{R_{1} d_{1}} + \denot{R_{2} d_{2}} \\
            & = & 3 * R_{1} + d_{1} + 3 * R_{2} + d_{2} \\
            & = & 3 * R_{1} +  3 * R_{2} + 3 * c + d \\
            & = & 3 * (R_{1} + R_{2} + c) + d \\
            & = & \denot{(R_{1} \oplus R_{2}) \oplus c) d}       
\end{eqnarray*}

Using similar arguments, we can show that the transition rules involving $\star$
also preserve denotational semantics.

\[
\infer[d' = neg(d)] 
  {neg(Ed) \rightarrow neg(E')d'}  
  {E \rightarrow E'}    
\]

\proof:
\begin{eqnarray*}
\denot{neg(Ed)}  
            & = & -\denot{Ed}  \\
            & = & -(3 * \denot{E} + d) \\
            & = & -(3 * \denot{E'} + d) \qquad \text{(By ref. trans.)}  \\
            & = & \denot{neg(E')d'}     
\end{eqnarray*}

\question{
\item[1.8] Show that every expression $\E$ reduces to a unique reduced numeral.
}

We showed in 1.6 that for every expression E, E has no transitions if it is a
reduced numeral. Thus, this is the terminal state in the operational semantics.
We observe that transition relation is \textit{deterministic}. Thus, we can
only arrive at a unique reduced numeral for a given expression $\E$.

An alternate argument is that we already shown the
appropriate \textit{denotational semantics} from balanced ternary expression to
integers. As each of these semantic clauses define the meaning of a compound
phrase from the meanings of its parts, these transition rules uniquely 
determine the semantic function on all expression arguments. 
Thus, every expression $\E$ reduces to a unique reduced numeral.

\question{
\item[1.9] Show that in your operational semantics, $\E \oplus \negation(E)$ and 
$\negation(E) \oplus E$ reduce to the numeral $\mathtt{0}$.
}
\begin{eqnarray*}
\denot{E \oplus \negation(E)}
	& = & \denot{E} + \denot{\negation(E)} \qquad \text{(By definition)} \\
	& = & \denot{E} - \denot{E} \\
	& = & 0	\\
	& = & \denot{\mathtt{0}} \qquad \text{(Assuming that this numeral denotes 0)}\\
\end{eqnarray*}
\begin{eqnarray*}
\denot{\negation(E) \oplus E} 
	& = & \denot{\negation(E)} + \denot{E}  \qquad \text{(By definition)} \\
	& = & \denot{E} - \denot{E} \\
	& = & 0	\\
	& = & \denot{\mathtt{0}}	\qquad \text{(Assuming that this numeral denotes 0)}\\
\end{eqnarray*}

Thus, $\E \oplus \negation(E)$ and $\negation(E) \oplus E$ reduce to the 
numeral $\mathtt{0}$.

\question{
\item[1.10] Consider the following alternative denotational semantics $\denot{\cdot}_a$ for ternary expressions:

\begin{align*}
& \denot{\mathtt{0}}_a = 0 \\ & \denot{\mathtt{+}}_a = 1 \\ & \denot{\mathtt{-}}_a = -1 \\
& \denot{\E \digit}_a = -3\denot{E}_a + \denot{\digit}_a \\
& \denot{\E_1 \oplus \E_2}_a = \denot{\E_1}_a + \denot{\E_2}_a \\
& \denot{\E_1 \star \E_2}_a = \denot{\E_1}_a * \denot{\E_2}_a \\
& \denot{\negation(\E)}_a = -\denot{\E}_a
\end{align*}

What is the range of this semantic function? Why? Define an operational semantics for 
ternary functions which is correct with respect to the above denotational semantics.
}

The range of this semantic function is $\mathbb{Z}$, i.e.
$$\denot{\cdot}_a : \textbf{Exp} \rightarrow \mathbb{Z}$$

The range of this function is $\mathbb{Z}$. We prove this by induction.

If depth is 0, by definition, $\denot{0} = 0$, $\denot{+} = +1$ and
$\denot{-} = -1$. Thus, these 3 integers are in the range of the function.

(a) Assume that integer $n$, where $n > +1$ is within the range of the function.
We show that then $n+1$ also is in the range of the function. 

Let $LST(n)$ denote the least significant trit and $OT(n)$ denote the
higher order trits excluding the least significant trit of the integer $n$.

If $LST(n) = -1$, then altering it to 0 gives as the higher integer. 
If $LST(n) = 0$, then altering it to +1 gives as the higher integer. 
If $LST(n) = +1$, then the next integer is of the form $(OT(n)-1)-1$.
As $OT(n)$ must be less than $n$, it must hold the induction hypothesis.
Thus, in all 3 possibilities, $n+1$ is also in the range of the function.

(b) By a symmetric argument, if integer $n$, where $n < -1$, is 
within the range of the function, we can show that $n-1$ also is in the range of
the function.

Thus, the range of this function is $\mathbb{Z}$ i.e. for every integer 
$i \in \mathbb{Z}$, there exists an expression $\E \in \textbf{Exp}$ such that
$\denot{\E} = i$.

\vspace{0.3in}
\textbf{Operational semantics :}
\vspace{0.3in}

\[
\infer[d \in \Digit]
  {0d \rightarrow d}
  {}   
\qquad
\infer[] 
  {neg(0) \rightarrow 0}  
  {}    
\qquad
\infer[] 
  {neg(+) \rightarrow -}  
  {}    
\qquad
\infer[] 
  {neg(-) \rightarrow +}  
  {}    
\]

\hrule

\[
\infer[d \in \Digit]
  {Ed \rightarrow E'd}
  {E \rightarrow E'}
\]

\[
\infer[]
  {E_{1} \oplus E_{2} \rightarrow E_{1}' \oplus E_{2}}
  {E_{1} \rightarrow E_{1}'} 
\qquad
\infer[]
  {R \oplus E_{2} \rightarrow R \oplus E_{2}'}
  {E_{2} \rightarrow E_{2}'}
\]

\[
\infer[]
  {E_{1} \star E_{2} \rightarrow E_{1}' \star E_{2}}
  {E_{1} \rightarrow E_{1}'} 
\qquad
\infer[d \in \Digit]
  {R \star E_{2} \rightarrow R \star E_{2}'}
  {E_{2} \rightarrow E_{2}'}
\]
\[
\infer[]
  {neg(E) \rightarrow neg(E')}
  {E \rightarrow E'}
\]

\hrule

\[
\infer[(c, d) = d_{1} +_{-3} d_{2}] 
  {d_{1} \oplus d_{2} \rightarrow cd}
  {}  
\]
\[
\infer[(c, d) = d_{1} +_{-3} d_{2}] 
  {d_{1} \oplus (R_{2} d_{2}) \rightarrow (c \oplus R_{2}) d}
  {}  
\]
\[
\infer[(c, d) = d_{1} +_{-3} d_{2}] 
  {(R_{1} d_{1}) \oplus d_{2}  \rightarrow (R_{1} \oplus c) d}
  {}  
\]
\[
\infer[(c, d) = d_{1} +_{-3} d_{2}] 
  {(R_{1} d_{1}) \oplus (R_{2} d_{2}) \rightarrow ((R_{1} + R_{2}) \oplus c) d}
  {}  
\]

\hrule

\[
\infer[(c, d) = d_{1} \star_{-3} d_{2}] 
  {d_{1} \star d_{2} \rightarrow cd}
  {}  
\]
\[
\infer[(c, d) = d_{1} \star_{-3} d_{2}] 
  {d_{1} \star (R_{2} d_{2}) \rightarrow (c \oplus R_{2}) d}
  {}  
\]
\[
\infer[(c, d) = d_{1} \star_{-3} d_{2}] 
  {(R_{1} d_{1}) \star d_{2}  \rightarrow (R_{1} \oplus c) d}
  {}  
\]
\[
\infer[(c, d) = d_{1} \star_{-3} d_{2}] 
  {(R_{1} d_{1}) \star (R_{2} d_{2}) \rightarrow ((R_{1} + R_{2}) \star c) d}
  {}  
\]

\hrule

\[
\infer[d' = neg(d)] 
  {neg(Ed) \rightarrow neg(E')d'}  
  {E \rightarrow E'}    
\]


Here, the operators are different compared to base-3 operators. For instance,

\begin{table}[h!]
    \centering
	
	$$ -_3 : \Digit \times \Digit \to \Digit \times \Digit$$ 

	\begin{tabular}{l | c c c}
	$-_3$ & - & 0 & + \\
	\hline
	- & (+, +) & (0, -) & (0, 0) \\
	0 & (0, -) & (0, 0) & (0, +) \\
	+ & (0, 0) & (0, +) & (-, -) \\
	\end{tabular}
\end{table}

\vspace{0.5in}

\question{
\item[1.11] In the alternative operational semantics you gave in 1.10, is it still true that every expression reduces to a 
unique reduced numeral? If so prove it, if not give a counterexample. 
}

Yes, ths argument we made in 1.8 still holds. For every expression $\E$, $\E$
has no transitions if it is a reduced numeral.
Thus, this is the terminal state in the alternative operational semantics. 
We observe that transition relation is deterministic. Thus, we can only arrive
at a unique reduced numeral for a given expression E.

An alternate argument is that we already shown the appropriate denotational semantics 
from balanced ternary expression to integers. As each of these semantic clauses
define the meaning of a compound phrase from the meanings of its parts, 
these transition rules uniquely determine the semantic function on all
expression arguments. Thus, every expression E reduces to a unique reduced numeral.

\end{enumerate}


%% ==================================================================
%% IMPERATIVE PROGRAMS
%% ==================================================================

\section{Imperative programs}

\question{
Recall the abstract syntax given in class for a very simple imperative programming language $\mathcal{L}$:

\begin{align*}
& \E := \mathtt{n} \; | \; \mathtt{x} \; | \; \E_1 \oplus \E_2 \; | \; \E_1 \star \E_2 \; | \; \negation(\E) \; | \; \ifsym \; \B \; \then \; \E_1 \; \elsesym \; \E_2 \\ 
& \B := \true \; | \; \false \; | \; \B_1 \; \andsym \; \B_2 \; | \; \B_1 \; \orsym \; \B_2 \; | \; \notsym \; \B \; | \; \E_1 \leq \E_2 \\
& \C := \skipsym \; | \; \mathtt{x} := \E \; | \; \C_1 ; \C_2 \; | \; \ifsym \; \B \; \then \; \C_1 \; \elsesym \; \C_2 \; | \; \whilesym \; \B \; \dosym \; \C
\end{align*}

where $\mathtt{n}$ ranges over the set $\mathbb{N}$ of natural numbers and $\mathtt{x}$ 
ranges over a fixed set of variables.
}

\begin{enumerate}

\question{
\item[2.1] When is $\C_1 ; \C_2$ the same as $\C_2 ; \C_1$ under the denotational semantics given in class?
}

We note that we use the partial function notation in this problem.\\

We derive the agreement property i.e. $\C_1 ; \C_2 \equiv_{st} \C_2 ; \C_1$
under the denotational semantics using structural induction on $\C$ based on the
definition of the semantic function $\C : St \rightharpoonup St$.

We first introduce these functions :

$\textbf{free} : \textbf{Exp}_{int} \rightarrow 2^{ \textbf{Ide} }$ \\
$\textbf{free} : \textbf{Exp}_{bool} \rightarrow 2^{ \textbf{Ide} }$ \\
$\textbf{reads} : \textbf{Com} \rightarrow 2^{ \textbf{Ide} }$ \\
$\textbf{writes} : \textbf{Com} \rightarrow 2^{ \textbf{Ide} }$. 

The first two functions, informally, map an integer expression and boolean
expression to the set of identifiers which occur free in that 
expression respectively. Formally,

\begin{align*}
& free(\mathtt{n}) = \{\} \\
& free(\mathtt{x}) = \{x\} \\
& free(\E_{1} \oplus \E_{2}) = free(\E_{1}) \cup free(\E_{2}) \\
& free(\E_{1} \star \E_{2}) = free(\E_{1}) \cup free(\E_{2}) \\
& free(neg(\E)) = free(\E) \\
& free(\ifsym \; \B \; \then \; \E_{1} \; \elsesym \; \E_{1}) = free(\B) \cup
free(\E_{1}) \cup free(\E_{2})
\end{align*}

\begin{align*}
& free(\true) = \{\} \\
& free(\false) = \{\} \\
& free(\B_{1} \; \andsym \; \B_{2}) = free(\B_{1}) \cup free(\B_{2}) \\
& free(\B_{1} \; \orsym \; \B_{2}) = free(\B_{1}) \cup free(\B_{2}) \\
& free(\notsym \; \B) = free(\B) \\
& free(\E_{1} \leq \E_{2}) = free(\E_{1}) \cup free(\E_{2})
\end{align*}

The latter two functions, informally, map a command $\C \in \textbf{Com}$ to the
set of free identifiers read and written by $\C$ respectively. Formally,

\begin{align*}
& reads(\skipsym) = \{\} \\
& reads(\mathtt{x} := \E) = free(\E) \\
& reads(\C_{1}; \C_{2}) = reads(\C_{1}) \cup reads(\C_{2}) \\
& reads(\ifsym \; \B \; \then \; \C_{1} \; \elsesym \; \C_{2}) = free(\B) \cup
reads(\C_{1}) \cup reads(\C_{2}) \\
& reads(\whilesym \; \B \; \dosym \; \C) = free(\B) \cup reads(\C)
\end{align*}

\begin{align*}
& writes(\skipsym) = \{\} \\
& writes(\mathtt{x} := \E) = \{\mathtt{x}\} \\
& writes(\C_{1}; \C_{2}) = writes(\C_{1}) \cup writes(\C_{2}) \\
& writes(\ifsym \; \B \; \then \; \C_{1} \; \elsesym \; \C_{2}) = 
writes(\C_{1}) \cup writes(\C_{2}) \\
& writes(\whilesym \; \B \; \dosym \; \C) = writes(\C)
\end{align*}  
  
Using these functions, we define a new function,  
$\textbf{conflict} : \textbf{Com} \times \textbf{Com} : \textbf{Bit}$.

It, informally, allows us to figure out if two commands conflict between their
read and write sets. Formally,
 
\begin{align*}
\textbf{conflict}(C_{1}, C_{2}) =
\{ \; \{ reads(C_{1}) \; \cap \; writes(C_{2} \} \; \cup \; \\
\{ writes(C_{1}) \; \cap \; reads(C_{2}) \} \; \cup \; \\
\{ writes(C_{1}) \; \cap \; writes(C_{2})\} \; \} \; \neq \; \phi)
\end{align*}  
  
With respect to data flow, $\C_1 ; \C_2 \equiv_{st} \C_2 ; \C_1$
if and only if $ \textbf{conflict}(C_{1}, C_{2}) = \false $

But, we also need to ensure that both commands terminate. Otherwise, if 
one of them is an infinite while loop, then the order of the commands
matters. Formally, $|\C_1| \neq \{\} $ and  $|\C_2| \neq \{\} $.
This ensures that they both have a terminal state.


\question{
\item[2.2] Express exponentiation of two natural numbers $\mathtt{x}$ and $\mathtt{y}$ as a program
in this language and prove its correctness with respect to the denotational semantics given in class.
}

We formulate the exponentiation program of two natural numbers $\mathtt{x}$ and
$\mathtt{y}$ in terms of a repeated multiplication of $\mathtt{x}$ with itself
for $\mathtt{y}$ times.

Thus, the following program satisfies our requirements (the curly brackets are
for clarity).

$$ \mathtt{r} := \mathtt{1}; \; \whilesym \; \mathtt{y} > \mathtt{0} \; \dosym
\; \{ \mathtt{r} := \mathtt{r} \star \mathtt{x} ; \; \mathtt{y} := \mathtt{y} -
1 \} $$

Here, the free identifier $\mathtt{r}$ computes the quantity of interest. Note
that $\mathtt{x} \geq \mathtt{1} $ and $\mathtt{y} \geq \mathtt{1}$. We prove
its correctness using induction on $\mathtt{y}$.

If $\mathtt{y} = \mathtt{1}$, the boolean expression in the $\whilesym$ command
is true exactly once thus the loop is executed exactly once. 
Thus, $\mathtt{r} = \mathtt{x} = \mathtt{x}^{\mathtt{1}} $.

Now, assume that the program is correct when $y = n$, where $n > 1$, i.e.
$\mathtt{r} = \mathtt{x}^{\mathtt{n}}$. Note that the $\whilesym$ command must have been 
triggered atleast twice in this case. 

Now, when $y = n + 1$, the loop must iterate atleast one more time. This is
because the boolean expression $\mathtt{y} > \mathtt{0}$ must
be $\true$ exactly one more time than the case when  $y = n$. We attribute
this to the fact that we decrement $\mathtt{y}$ by one in each iteration.
Thus, in that extra iteration, the multiplication with $\mathtt{x}$ results in
$\mathtt{r} = \mathtt{x}^{\mathtt{n}}  * \mathtt{x} =
\mathtt{x}^{\mathtt{n+1}}$. 
We also note that this program must terminate for all pairs of finite natural
numbers $\mathtt{x}$ and $\mathtt{y}$. Thus, we proved the correctness of the 
program by induction.

\question{
Consider a programming language $\mathcal{L}_e$ extended as follows:

\begin{align*}
& \E := \mathtt{n} \; | \; \mathtt{x} \; | \; \E_1 \oplus \E_2 \; | \; \E_1 \star \E_2 \; | \; \negation(\E) \; | \; \ifsym \; \B \; \then \; \E_1 \; \elsesym \; \E_2 \\
& \B := \true \; | \; \false \; | \; \B_1 \; \andsym \; \B_2 \; | \; \B_1 \; \orsym \; \B_2 \; | \; \notsym \; \B \; | \; \E_1 \leq \E_2 \\
& {\color{red} \LL := \nil \; | \; (\mathtt{n},\C)::\LL} \\
& \C := \skipsym \; | \; \mathtt{x} := \E \; | \; \C_1 ; \C_2 \; | \; \ifsym \; \B \; \then \; \C_1 \; \elsesym \; \C_2 \; | \; \whilesym \; \B \; \dosym \; \C \; | \; {\color{red} \case \; \E \; \LL \; \C}
\end{align*}

The command $\case \; \E \; \LL \; \C$ evaluates $\E$ to obtain a natural number $\mathtt{n}$,
identifies the first pair $(\mathtt{n},\C_1)$ which occurs in the list $\LL$, and performs $\C_1$;
if no such pair exists, it performs $\C$.
}

\question{
\item[2.3] Define an appropriate denotational semantics for this extended programming language. 
You only need to specify the evaluation function on the new constructs.
}

The new construct is the case command and the list primitive.
We extend the semantic function $\C : St \rightharpoonup St$ to handle this
command.

Before that, we introduce two helper functions for accessing list.

$\textbf{exists} : \textbf{List} \times \mathbb{N} \rightarrow \textbf{Bit}$ \\
$\textbf{get} : \textbf{List} \times \mathbb{N} \rightarrow \textbf{Com}$ \\ 

Informally, the first command allows us to check if the list has a pair
$(\mathtt{n},\C)$ given a number $n \in \mathbb{N}$. Formally,
\begin{align*}
& exists(\LL, \mathtt{n}) = \true  \; \; \ifsym \; \exists \; (\mathtt{n},\C)
\in
\LL
\\
& exists(\LL, \mathtt{n}) = \false \; , \; \text{otherwise} 
\end{align*}  

Informally, the later command allows us to get the relevant command $\C$ if the
list has a pair $(\mathtt{n},\C)$ given a number $n \in \mathbb{N}$. Formally,

\begin{align*}
& get(\LL, \mathtt{n}) = \C  \; \; \ifsym \; \exists \; (\mathtt{n},\C) \in
\LL
\\
& get(\LL, \mathtt{n}) = \skipsym \; , \; \text{otherwise} 
\end{align*}  

Now, we can extend the semantic function thus:

$\; | \; \case \; \E \; \LL \; \C \; |\; = 
\C \; \circ \; | exists(\LL, |\E|)|_{\false} \cup \; 
get(\LL, \mathtt{n}) \; \circ \; |exists(\LL, |\E|)|_{\true} $

\question{
\item[2.4] Show that any program written in the extended programming language $\mathcal{L}_e$ 
can be written in $\mathcal{L}$, by constructing an appropriate translation function.
}

We observe that the $\case$ command can be unrolled into multiple $\ifsym \; \B
\; \then \; \C_1 \; \elsesym \; \C_2 \;$ statements, one for each number in the
list $\LL$.

Before we define the translation function, we introduce a helper function
for accessing the first list element.

$\textbf{head} : \textbf{List} \rightarrow \mathbb{N} \times \textbf{Com} $ \\

Informally, it allows us to retrieve an element from the list. Formally,

\begin{align*}
& elem(\LL) = (\mathtt{n}, \C_) \; \; \ifsym \; \exists \; (\mathtt{n},\C)  \in
\LL\\
& elem(\LL) = (0, \skipsym) \; \text{otherwise} 
\end{align*}  

Now, we can define the translation function thus:
\begin{align*}
& \; | \; \case \; \E \; \LL \; \C \; |\; = 
\ifsym \; elem(\LL) \neq (0, \skipsym)  \; 
\then \; \C_1 \; \elsesym \; \C \; 
| \; \case \; \E \; \LL' \; \C \; \\
& \text{where} (\mathtt{n},\C_1) \in \; \LL  \; \ifsym \; elem(\LL) = (0,
\skipsym) \\
& \text{and} \; \; \LL' = \LL - elem(\LL)
\end{align*} 

This allows us to unroll the $\case$ command into multiple $\ifsym \; \B \;
\then \; \C_1 \; \elsesym \; \C_2 \;$ statements.

\question{
\item[2.5] Show that the translation function you constructed in 2.3 preserves denotational semantics.
}

\question{
\item[2.6] How do your answers for 2.4 and 2.5 change if we change the abstract syntax for $\LL$ to
\[ \LL := \nil \; | \; ({\color{red} \E},\C)::\LL \]
instead?
}

\end{enumerate}

\end{document}


